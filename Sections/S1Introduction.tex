\section{Introduction}
%In 2008, the first ever cryptocurrency Bitcoin, solved the problem of double spending by using 

\subsection{What is a blockchain?}

A blockchain is a growing list of records, called blocks, which are linked using cryptography. Each block contains a cryptographic hash of the previous block, a timestamp, and transaction data (generally represented as a merkle tree root hash).

By design, a blockchain is resistant to modification of the data. It is "an open, distributed ledger that can record transactions between two parties efficiently and in a verifiable and permanent way". For use as a distributed ledger, a blockchain is typically managed by a peer-to-peer network collectively adhering to a protocol for inter-node communication and validating new blocks. Once recorded, the data in any given block cannot be altered retroactively without alteration of all subsequent blocks, which requires consensus of the network majority. Although blockchain records are not unalterable, blockchains may be considered secure by design and exemplify a distributed computing system with high Byzantine fault tolerance. Decentralized consensus has therefore been claimed with a blockchain.

\subsection{Problem Statement}

A blockchain system's throughput depends on the number of nodes participating in the validation of a transaction. More the number of nodes means more resistance to faults but resulting in higher latency and lower throughputs. The objective of this project is to propose a blockchain system where the number of validators is determined by the market dynamics.
